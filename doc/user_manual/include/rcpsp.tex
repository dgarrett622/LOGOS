\section{Resource-Constrained Project Scheduling Problem (RCPSP)}
\label{sec:RCPSP}

In this section, we consider the resource-constrained project scheduling problem (RCPSP).
In RCPSP, the activities of a project have to be scheduled such that the makespan of the
project is minimized. Thereby, technological precedence constraints have to be observed
as well as limited capacities of the renewable resources that are required to accomplish
the activities. The RCPSP for the scheduling of maintenance and surveillance activities
of nuclear power plant can be summarized as follows:

We consider a project that consists of a set of $J$ jobs (or tasks). Due to technological
requirements, precedence relations among some of the jobs enforce that job $j=2,3,\cdots,J$
may not be started before all its predecessors, denoted by $P_j$, are finished. Here,
$j=1$ indexes an artificial job with zero duration, which precedes all jobs that can
start at time zero, and  $j=J$ indexes an artificial final job, again with zero duration,
which represents the end of the project. Executing job $j$ takes $d_j$ time periods and
is supported by a set, $R$, of renewable resources. Consider a horizon with an upper
bound, $T$, on the project's makespan, i.e., the time at which the final job
is completed. We assume $K_{r}^{p}$ units of renewable resource $r\in R$, are available
in time period $t=1,2,\cdots,T$. Job $j$ requires $k_{jr}^{p}$  units of the renewable
resource, $r\in R$, for each period of the job’s duration, i.e., for time periods
when the job is in process.

The objective is to find a schedule which minimizes the project’s makespan while
respecting the constraints imposed by the precedence relations and the limited
resource availability.

\[
\begin{array}{ll}
%%%%%%%%%%%%%% INDICES AND Parameters %%%%%%%%%%%%%%%%
\multicolumn{2}{l}{\mbox{\em Indexes and parameters:} } \\
t = 1, 2, \cdots, T & \mbox{time periods, where $T$ is an upper bound on the projects's makespan} \\
j = 1, 2, \cdots, J & \mbox{jobs, with $j=1$ and $j=J$ denoting artificial jobs} \\
r\in R & \mbox{set of renewable resources} \\
d_j & \mbox{duration of job $j$} \\
K_{r}^{p} & \mbox{number of units of renewable resource $r$ available in period $t$} \\
k_{jr}^{p} & \mbox{number of units of renewable resource $r$ consumed by job $j$ while in process} \\
P_j & \mbox{set of immediate predecessors of job j} \\
\\
%%%%%%%%%%%%%% Decision Variables %%%%%%%%%%%%%%%%
\multicolumn{2}{l}{\mbox{\em Decision variables:}} \\
x_{jt} & \mbox{1 if job $j$ completes in period $t$; 0 otherwise} \hspace*{4.0in}\\
\end{array}
\]

\vst \noi {\em Formulation:}
\begin{subequations}\label{rcpsp_eq}
\begin{eqnarray}
&\dst \min &  \dst \sum_{t=1}^{T} (t-1) x_{Jt}  \\
& s. t.  & \sum_{t=1}^{T} x_{jt} = 1,   j=1, \cdots, J \\
& & \sum_{t' \le t} x_{jt'} \le \sum_{t' \le t - d_j} x_{it'}, j = 2, \cdots, J; t = 1, \cdots, T; i\in P_j \\
& & \sum_{j=1}^{J} \sum_{t'=t}^{t+d_j-1} k_{jr}^{p} x_{jt'} \le K_{r}^{p}, r \in R; t=1, \cdots, T \\
& & x_{jt} \in \{0,1\}, j=1, \cdots, J, t = 1, \cdots, T
\end{eqnarray}
\end{subequations}

The RCPSP model must consists the following modeling components: \xmlNode{Sets},
\xmlNode{Parameters}, and \xmlNode{Settings}. Each of these components is illustrated
in the following sections.

\subsection{Sets}
\label{subsec:rcpsp_sets}

This subsection contains information regarding the XML nodes used to define the
\xmlNode{Sets} of the RCPSP model being performed through LOGOS.
\xmlNode{Sets} specifies a collection of data, possibly including
numeric data (e.g. real or integer values) as well as symbolic data (e.g. strings)
typically used to specify the valid indices for indexed components.
\nb Numeric data provided in \xmlNode{Sets} would be treated as strings.
\xmlNode{Sets} accepts the following additional sub-nodes:
\begin{itemize}
  \item \xmlNode{tasks}, \xmlDesc{comma/space-separated string, required}, specifies
  the valid indices for tasks.
  \item \xmlNode{resources}, \xmlDesc{comma/space-separated string, required},
  specifies the indices for renewable resources.
  \item \xmlNode{predecessors}, \xmlDesc{comma/space-separated string, required},
  specifies the indices for preceding tasks.
  \item \xmlNode{successors}, \xmlDesc{comma/space-separated string, required},
  specifies indices for successors.
  This sub-node accepts the following attribute:
  \begin{itemize}
    \item \xmlAttr{index}, \xmlDesc{string, required}, specifies the index dependence.
    Valid index is \xmlString{predecessors}.
  \end{itemize}
\end{itemize}

Example XML:
\begin{lstlisting}[style=XML]
<Sets>
  <tasks>
    1, 2, 3, 4, 5, 6, 7, 8, 9, 10, 11, 12
  </tasks>

  <resources>
    r1 r2 r3 r4
  </resources>

  <predecessors>
    1, 2, 3, 4, 5, 6, 7, 8, 9, 10, 11
  </predecessors>

  <successors index='predecessors'>
    s1 s2 s3 s4;
    s1;
    s1;
    s1;
    s1;
    s1;
    s1;
    s1;
    s1;
    s1;
    s1
  </successors>
</Sets>
\end{lstlisting}


\subsection{Parameters}
\label{subsec:rcpsp_params}

This

\subsection{Settings}
\label{subsec:rcpsp_settings}

This



Example LOGOS input XML for RCPSP Model:
\begin{lstlisting}[style=XML]
<?xml version="1.0" encoding="UTF-8"?>
<Logos>
  <Sets>
    <tasks>
      1, 2, 3, 4, 5, 6, 7, 8, 9, 10, 11, 12
    </tasks>

    <resources>
      r1 r2 r3 r4
    </resources>

    <predecessors>
      1, 2, 3, 4, 5, 6, 7, 8, 9, 10, 11
    </predecessors>

    <successors index='predecessors'>
      s1 s2 s3 s4;
      s1;
      s1;
      s1;
      s1;
      s1;
      s1;
      s1;
      s1;
      s1;
      s1
    </successors>
  </Sets>

  <Parameters>
    <available_resources index='resources'>
      13 13 13 12
    </available_resources>
    <task_resource_consumption index='tasks, resources'>
      0	0	0	0
      10	0	0	0
      0	7	0	0
      0	9	0	0
      0	4	0	0
      0	0	0	6
      10	0	0	0
      0	0	6	0
      0	0	0	8
      0	6	0	0
      0	0	0	5
      0	0	0	0
    </task_resource_consumption>

    <task_duration index="tasks">
      0
      8
      1
      10
      6
      5
      8
      9
      1
      9
      8
      0
    </task_duration>

    <task_successors index='successors' type='str'>
      2 3 4 9
      5
      7
      8
      6
      10
      11
      10
      12
      9
      12
    </task_successors>
  </Parameters>

  <Settings>
    <makespan_upperbound>65</makespan_upperbound>
    <solver>cbc</solver>
    <sense>minimize</sense>
    <problem_type>rcpsp</problem_type>
  </Settings>
</Logos>
\end{lstlisting}
