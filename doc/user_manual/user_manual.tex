%
% This is an example LaTeX file which uses the SANDreport class file.
% It shows how a SAND report should be formatted, what sections and
% elements it should contain, and how to use the SANDreport class.
% It uses the LaTeX article class, but not the strict option.
% ItINLreport uses .eps logos and files to show how pdflatex can be used
%
% Get the latest version of the class file and more at
%    http://www.cs.sandia.gov/~rolf/SANDreport
%
% This file and the SANDreport.cls file are based on information
% contained in "Guide to Preparing {SAND} Reports", Sand98-0730, edited
% by Tamara K. Locke, and the newer "Guide to Preparing SAND Reports and
% Other Communication Products", SAND2002-2068P.
% Please send corrections and suggestions for improvements to
% Rolf Riesen, Org. 9223, MS 1110, rolf@cs.sandia.gov
%

\documentclass[pdf,12pt]{INLreport}

\newcommand{\vst}{\vspace*{0.2in}}
\newcommand{\dst}{\displaystyle}
\newcommand{\noi}{\noindent}

% pslatex is really old (1994).  It attempts to merge the times and mathptm packages.
% My opinion is that it produces a really bad looking math font.  So why are we using it?
% If you just want to change the text font, you should just \usepackage{times}.
% \usepackage{pslatex}
\usepackage{times}
\usepackage[FIGBOTCAP,normal,bf,tight]{subfigure}
\usepackage{amsmath}
\usepackage{amssymb}
\usepackage{soul}
\usepackage{pifont}
\usepackage{enumerate}
\usepackage{listings}
\usepackage{fullpage}
\usepackage{xcolor}          % Using xcolor for more robust color specification
\usepackage{ifthen}          % For simple checking in newcommand blocks
\usepackage{textcomp}
\usepackage{mathtools}
%\usepackage{relsize}
\usepackage{lscape}
\usepackage[toc,page]{appendix}

\graphicspath{{./figures/}}

\newtheorem{mydef}{Definition}
\newcommand{\norm}[1]{\lVert#1\rVert}
%\usepackage[table,xcdraw]{xcolor}
%\usepackage{authblk}         % For making the author list look prettier
%\renewcommand\Authsep{,~\,}

% Custom colors
\definecolor{deepblue}{rgb}{0,0,0.5}
\definecolor{deepred}{rgb}{0.6,0,0}
\definecolor{deepgreen}{rgb}{0,0.5,0}
\definecolor{forestgreen}{RGB}{34,139,34}
\definecolor{orangered}{RGB}{239,134,64}
\definecolor{darkblue}{rgb}{0.0,0.0,0.6}
\definecolor{gray}{rgb}{0.4,0.4,0.4}

\lstset {
  basicstyle=\ttfamily,
  frame=single
}


\setcounter{secnumdepth}{5}
\lstdefinestyle{XML} {
    language=XML,
    extendedchars=true,
    breaklines=true,
    breakatwhitespace=true,
%    emph={name,dim,interactive,overwrite},
    emphstyle=\color{red},
    basicstyle=\ttfamily,
%    columns=fullflexible,
    commentstyle=\color{gray}\upshape,
    morestring=[b]",
    morecomment=[s]{<?}{?>},
    morecomment=[s][\color{forestgreen}]{<!--}{-->},
    keywordstyle=\color{cyan},
    stringstyle=\ttfamily\color{black},
    tagstyle=\color{darkblue}\bf\ttfamily,
    morekeywords={name,type},
%    morekeywords={name,attribute,source,variables,version,type,release,x,z,y,xlabel,ylabel,how,text,param1,param2,color,label},
}
\lstset{language=python,upquote=true}

\usepackage{titlesec}
\newcommand{\sectionbreak}{\clearpage}
\setcounter{secnumdepth}{4}

%\titleformat{\paragraph}
%{\normalfont\normalsize\bfseries}{\theparagraph}{1em}{}
%\titlespacing*{\paragraph}
%{0pt}{3.25ex plus 1ex minus .2ex}{1.5ex plus .2ex}

%%%%%%%% Begin comands definition to input python code into document
\usepackage[utf8]{inputenc}

% Default fixed font does not support bold face
\DeclareFixedFont{\ttb}{T1}{txtt}{bx}{n}{9} % for bold
\DeclareFixedFont{\ttm}{T1}{txtt}{m}{n}{9}  % for normal

\usepackage{listings}

% Python style for highlighting
\newcommand\pythonstyle{\lstset{
language=Python,
basicstyle=\ttm,
otherkeywords={self, none, return},             % Add keywords here
keywordstyle=\ttb\color{deepblue},
emph={MyClass,__init__},          % Custom highlighting
emphstyle=\ttb\color{deepred},    % Custom highlighting style
stringstyle=\color{deepgreen},
frame=tb,                         % Any extra options here
showstringspaces=false            %
}}


% Python environment
\lstnewenvironment{python}[1][]
{
\pythonstyle
\lstset{#1}
}
{}

% Python for external files
\newcommand\pythonexternal[2][]{{
\pythonstyle
\lstinputlisting[#1]{#2}}}

\lstnewenvironment{xml}
{}
{}

% Python for inline
\newcommand\pythoninline[1]{{\pythonstyle\lstinline!#1!}}


\def\DRAFT{} % Uncomment this if you want to see the notes people have been adding
% Comment command for developers (Should only be used under active development)
\ifdefined\DRAFT
  \newcommand{\nameLabeler}[3]{\textcolor{#2}{[[#1: #3]]}}
\else
  \newcommand{\nameLabeler}[3]{}
\fi
% Commands for making the LaTeX a bit more uniform and cleaner
\newcommand{\TODO}[1]    {\textcolor{red}{\textit{(#1)}}}
\newcommand{\xmlAttrRequired}[1] {\textcolor{red}{\textbf{\texttt{#1}}}}
\newcommand{\xmlAttr}[1] {\textcolor{cyan}{\textbf{\texttt{#1}}}}
\newcommand{\xmlNodeRequired}[1] {\textcolor{deepblue}{\textbf{\texttt{<#1>}}}}
\newcommand{\xmlNode}[1] {\textcolor{darkblue}{\textbf{\texttt{<#1>}}}}
\newcommand{\xmlString}[1] {\textcolor{black}{\textbf{\texttt{'#1'}}}}
\newcommand{\xmlDesc}[1] {\textbf{\textit{#1}}} % Maybe a misnomer, but I am
                                                % using this to detail the data
                                                % type and necessity of an XML
                                                % node or attribute,
                                                % xmlDesc = XML description
\newcommand{\default}[1]{~\\*\textit{Default: #1}}
\newcommand{\nb} {\textcolor{deepgreen}{\textbf{~Note:}}~}


%%%%%%%% End comands definition to input python code into document

%\usepackage[dvips,light,first,bottomafter]{draftcopy}
%\draftcopyName{Sample, contains no OUO}{70}
%\draftcopyName{Draft}{300}

% The bm package provides \bm for bold math fonts.  Apparently
% \boldsymbol, which I used to always use, is now considered
% obsolete.  Also, \boldsymbol doesn't even seem to work with
% the fonts used in this particular document...
\usepackage{bm}


% Define tensors to be in bold math font.
\newcommand{\tensor}[1]{{\bm{#1}}}

% Override the formatting used by \vec.  Instead of a little arrow
% over the letter, this creates a bold character.
\renewcommand{\vec}{\bm}

% Define unit vector notation.  If you don't override the
% behavior of \vec, you probably want to use the second one.
\newcommand{\unit}[1]{\hat{\bm{#1}}}
% \newcommand{\unit}[1]{\hat{#1}}

% Use this to refer to a single component of a unit vector.
\newcommand{\scalarunit}[1]{\hat{#1}}

% \toprule, \midrule, \bottomrule for tables
\usepackage{booktabs}

% \llbracket, \rrbracket
\usepackage{stmaryrd}

\usepackage{hyperref}
\hypersetup{
    colorlinks,
    citecolor=black,
    filecolor=black,
    linkcolor=black,
    urlcolor=black
}

% Compress lists of citations like [33,34,35,36,37] to [33-37]
\usepackage{cite}

% If you want to relax some of the SAND98-0730 requirements, use the "relax"
% option. It adds spaces and boldface in the table of contents, and does not
% force the page layout sizes.
% e.g. \documentclass[relax,12pt]{SANDreport}
%
% You can also use the "strict" option, which applies even more of the
% SAND98-0730 guidelines. It gets rid of section numbers which are often
% useful; e.g. \documentclass[strict]{SANDreport}

% The INLreport class uses \flushbottom formatting by default (since
% it's intended to be two-sided document).  \flushbottom causes
% additional space to be inserted both before and after paragraphs so
% that no matter how much text is actually available, it fills up the
% page from top to bottom.  My feeling is that \raggedbottom looks much
% better, primarily because most people will view the report
% electronically and not in a two-sided printed format where some argue
% \raggedbottom looks worse.  If we really want to have the original
% behavior, we can comment out this line...
\raggedbottom
\setcounter{secnumdepth}{5} % show 5 levels of subsection
\setcounter{tocdepth}{5} % include 5 levels of subsection in table of contents

% ---------------------------------------------------------------------------- %
%
% Set the title, author, and date
%
\title{LOGOS User Manual}
%\author{%
%\begin{tabular}{c} Author 1 \\ University1 \\ Mail1 \\ \\
%Author 3 \\ University3 \\ Mail3 \end{tabular} \and
%\begin{tabular}{c} Author 2 \\ University2 \\ Mail2 \\ \\
%Author 4 \\ University4 \\ Mail4\\
%\end{tabular} }


\author{
\\Congjian Wang
\\Diego Mandelli
}

% There is a "Printed" date on the title page of a SAND report, so
% the generic \date should [WorkingDir:]generally be empty.
\date{}


% ---------------------------------------------------------------------------- %
% Set some things we need for SAND reports. These are mandatory
%
\SANDnum{INL/EXT-20-61006}
\SANDprintDate{\today}
\SANDauthor{Congjian Wang and Diego Mandelli}
\SANDreleaseType{Revision 0}
\def\component#1{\texttt{#1}}

% ---------------------------------------------------------------------------- %
\newcommand{\systemtau}{\tensor{\tau}_{\!\text{SUPG}}}

\usepackage{placeins}
\usepackage{array}

\newcolumntype{L}[1]{>{\raggedright\let\newline\\\arraybackslash\hspace{0pt}}m{#1}}
\newcolumntype{C}[1]{>{\centering\let\newline\\\arraybackslash\hspace{0pt}}m{#1}}
\newcolumntype{R}[1]{>{\raggedleft\let\newline\\\arraybackslash\hspace{0pt}}m{#1}}

% ---------------------------------------------------------------------------- %
%
% Start the document
%
\begin{document}
    \maketitle

    % ------------------------------------------------------------------------ %
    % The table of contents and list of figures and tables
    % Comment out \listoffigures and \listoftables if there are no
    % figures or tables. Make sure this starts on an odd numbered page
    %
    \cleardoublepage		% TOC needs to start on an odd page
    \tableofcontents
    %\listoffigures
    %\listoftables
    % ---------------------------------------------------------------------- %
    \SANDmain

    % ---------------------------------------------------------------------- %
    % This is where the body of the report begins; usually with an Introduction
    %
    \section{Introduction}
\label{sec:Introduction}

Industry equipment reliability (ER) and asset management (AM) programs are essential elements that
help ensure the safe and economical operation of nuclear power plants (NPPs).
The effectiveness of
these programs is addressed in several industry-developed and regulatory programs.
However, these programs have proven labor-intensive and expensive.
There is an opportunity to significantly
enhance the collection, analysis, and use of this information in order to provide more cost-effective plant
operation.
LOGOS provides computational capabilities to optimize plant resources such as
maintenance optimization (ER application) and optimal component replacement schedule (AM application)
by using state-of-the-art discrete optimization methods.

LOGOS is a software package and RAVEN~\cite{RAVEN,RAVENtheoryMan} plugin that
contains a set of discrete optimization models designed to solve capital budgeting optimization
problems by integrating economic and reliability data into the analysis framework.
More specifically, given system, structure and component
(SSC) health (e.g., failure rate or probability), O\&M costs, replacement costs, and cost
associated with component failure and budget constraints, LOGOS provides the optimal set of projects
(e.g., SSC replacement) to maximize profit and satisfy the provided reliability requirements.
The aforementioned input data can be either deterministic or stochastic in nature (i.e., they can be point values or probability distribution functions).
In the latter case, several scenarios are generated by
sampling the provided distributions.

The developed models are based on different versions of the knapsack optimization problem.
Two main classes of optimization models were initially developed: deterministic and stochastic.
Stochastic optimization models evolve deterministic models by explicitly considering data
uncertainties associated with constraints or item cost and reward. In FY-20, we moved forward
by implementing two schedule optimization methods. The first one reformulates the
capital budgeting problem in a distributionally robust form which allows the user to rely on
data directly rather than proposing a distribution from the data itself. The second one
reformulates the capital budgeting explicitly using risk measures as variable to be maximized or
minimized.

These models can be employed as stand-alone models or interfaced with the INL-developed RAVEN code
to propagate data uncertainties and analyze the generated data (i.e., sensitivity analysis).

\subsection{Acquiring and Installing LOGOS}
LOGOS is supported on three separate computing platforms: Linux, OSX (Apple Macintosh), and Microsoft
Windows.
 Currently, LOGOS can be downloaded from the LOGOS GitLab repository:
\url{https://hpcgitlab.hpc.inl.gov/RAVEN_PLUGINS/LOGOS.git}.
New users should contact LOGOS developers to get started with LOGOS.
This typically involves the following steps:

\begin{itemize}
  \item \textit{Download LOGOS}
    \\ You can download the source code for LOGOS from \url{https://hpcgitlab.hpc.inl.gov/RAVEN_PLUGINS/LOGOS.git}.
  \item \textit{Install LOGOS dependencies}
	\begin{lstlisting}[language=bash]
	path/to/LOGOS/build.sh --install
	\end{lstlisting}
  \item \textit{Activate LOGOS Libraries}
  \begin{lstlisting}[language=bash]
  source activate LOGOS_libraries
  \end{lstlisting}
  \item \textit{Test LOGOS}
	\begin{lstlisting}[language=bash]
	python run_tests.py
	\end{lstlisting}
  	Alternatively, the \texttt{logos} script
    contained in the folder ``\texttt{LOGOS}'' can be directly used:
\begin{lstlisting}[language=bash]
path/to/LOGOS/logos -i <inputFile.xml> -o <outputFile.csv>
\end{lstlisting}
	\item \textit{For use as a RAVEN Plugin}, RAVEN must first be downloaded from
  \url{https://github.com/idaholab/raven.git}.
		\\ Detailed instructions are available from \url{https://github.com/idaholab/raven/wiki}.
    To register a plugin with RAVEN and make its components accessible, run the script:
    \begin{lstlisting}[language=bash]
raven/scripts/install_plugins.py -s /abs/path/to/LOGOS
  	\end{lstlisting}
    After the plugin registration, then following the installation instruction at
    \url{https://github.com/idaholab/raven/wiki/installationMain} to install the
    required dependencies.
\end{itemize}

\subsection{User Manual Formats}
In this manual, we employ the following formats to highlight specific elements with
particular meanings (i.e., input structure, examples, and terminal commands):

\begin{itemize}
\item \textbf{\textit{Python Coding:}}
\begin{lstlisting}[language=python]
class AClass():
  def aMethodImplementation(self):
    pass
\end{lstlisting}
\item \textbf{\textit{LOGOS XML input example:}}
\begin{lstlisting}[style=XML,morekeywords={anAttribute}]
<MainXMLBlock>
  ...
  <aXMLnode anAttribute='aValue'>
     <aSubNode>body</aSubNode>
  </aXMLnode>
  <!-- This is  commented block -->
  ...
</MainXMLBlock>
\end{lstlisting}
\item \textbf{\textit{Bash Commands:}}
\begin{lstlisting}[language=bash]
cd path/to/LOGOS/
./build.sh --install
cd ../../
\end{lstlisting}
\end{itemize}

\subsection{Components of LOGOS}
In LOGOS, eXtensible Markup Language (XML) format is used to create the input file.
For more information about XML, please click on the link:
\href{https://www.w3schools.com/xml/default.asp}{\textbf{XML tutorial}}.
%
\\The main input blocks are as follows:
\begin{itemize}
  \item \xmlNode{Logos}: The root node containing the
  entire input; all of
  the subsequent blocks fit inside the \emph{LOGOS} block.
  %
  \item \xmlNode{Settings}: Specifies the calculation settings (i.e. options for
	optimization solvers, options for constraints, and working directory.)
  %
  \item \xmlNode{Sets}: Specifies a collection of data, possibly including
	numeric data (e.g. real or integer values) as well as symbolic data (e.g. strings)
	typically used to specify valid indices for indexed components.
	\nb numeric data provided in the \xmlNode{Sets} would be treated as strings.
  %
	\item \xmlNode{Parameters}: Specifies a collection of parameters, which are
  numerical values used to formulate constraints and objectives in a
	optimization model. A parameter can denote a single value, an array of values, or a multi-dimensional
	array of values.
	%
	\item \xmlNode{Uncertainties}: Specifies a collection of scenarios, which are
	numerical values used to simulate variations within parameters. A scenarios should follow
	the same format as the parameter.
	%
	%
	\item \xmlNode{ExternalConstraints}: Specifies a collection of external constraints, which are
  Python functions used to add additional constraints to the
	current optimization problem.
	%
\end{itemize}

Each of these components is explained in dedicated sections of the user manual.

\subsection{Capabilities of LOGOS}
This document provides a detailed description of LOGOS.
The features included in LOGOS are:
\begin{itemize}
    \item Overview of modeling components (see Section~\ref{sec:ModelingComponents})
	\item Deterministic capital budgeting (see Section~\ref{sec:DeterministicCapitalBudgeting})
	\item Prioritizing project selection to hedge against uncertainty (see Section~\ref{sec:StochasticCapitalBudgeting})
    \item Distributionally robust optimization (see Section~\ref{sec:DROCapitalBudgeting})
    \item Risk-based stochastic capital budgeting using conditional Value-at-Risk (See Section~\ref{sec:CVaR})
    \item Plugin for the RAVEN code (see Section~\ref{sec:RavenPlugin})
	\item SSC cashflow and NPV models (see Section~\ref{sec:SSCNPV})
\end{itemize}

    \section{Overview of Capital Budgeting Modeling Components}
\label{sec:ModelingComponents}

We consider a capital budgeting problem for a nuclear generation station, with possible
extension to a larger fleet of plants.
Due to limited resources, we can only select a
subset from a number of candidate investment projects.
Our goal is to maximize overall net
present value (NPV), or a variant of this objective, when we incorporate uncertainty into
project cost and revenue streams.
In doing so, we must respect resource limits
and capture key structural and stochastic dependencies of the system.
Example projects include upgrading a steam turbine, refurbishing or replacing a set of reactor coolant pumps,
and replacing a set of feed-water heaters.
Selecting an individual project has multiple facets and implications.

\begin{itemize}
  \item \textbf{Rewards or Net Present Values}: Selecting a project can improve revenue (e.g.,
  upgrading a steam turbine may lead to an uprate in plant capacity resulting in larger
  revenue from selling power.) Replacing a key system component can improve reliability,
  increasing revenue due to a reduction in forced outages as well as operations and
  maintenance (O\&M) costs. Choosing to perform minimum maintenance versus refurbishing
  a component or replacing and improving a system can produce reward streams that
  can be negative or positive depending on the selection. Parameter
  \xmlNode{net\_present\_values} is used to specify the rewards (see Section~\ref{subsec:Parameters}).

  \item \textbf{Resources and Liabilities}: Critical resources, including (i) capital costs,
  (ii) O\&M costs, (iii) time and labor-hours during a planned outage, and (iv) personnel,
  installation and maintenance of equipment, workspaces, etc.. Within these categories, resources
  can further sub-categorized, (witch each subcategory having its own budget), according to the plant’s organizational
  structure to provide multiple “colors” of money within capital costs, O\&M costs,
  personnel availability, etc.. The set \xmlNode{resources} and parameter
  \xmlNode{available\_capitals} are used to specify the resources and
  liabilities (see Section~\ref{subsec:Sets} and Section~\ref{subsec:Parameters}).

  \item \textbf{Costs}: Selecting a project in year $t$ induces multiple
  cost streams in year $t$ as well as in subsequent years; we interpret “cost” broadly to
  include commitment of critical resources. The parameter \xmlNode{costs} is used to specify
  the costs (see Section~\ref{subsec:Parameters}).

  \item \textbf{Time Periods}: Multiple capital projects can compete for same time period,
  limiting project selection. The set \xmlNode{time\_periods} is used to provide indices for
  \textbf{costs} and \textbf{available capitals} (see Section~\ref{subsec:Sets}).

  \item \textbf{Options}: The goal of selecting a project is typically to improve or maintain
  a particular function the plant performs, and there may be multiple ways to carry out
  the task. A project may be performed over a three-year period--say, years `$t$, $t+1$, $t+2$'--or the
  start of the project could instead be two years hence, changing the equation to
  `$t+2$, $t+3$, $t+4$'. Alternatively, at increased cost and benefit, it may be
  possible to complete the project in two years: `$t$, $t+1$' or `$t+2$, $t+3$'. When selecting a project
  to uprate plant capacity, we may have the options of increasing it by 3\% or 6\%.
  In all these cases, we can perform the project in, at most, one particular way, out of a collection of
  options. We represent this by cloning a “project” into multiple project-option pairs,
  and adding a constraint saying that we can select, at most, one from this set of options.
  The set \xmlNode{options} is used to provided indices for these multiple project-option pairs
  (see Section~\ref{subsec:Sets}).

  \item \textbf{Capitals}: If we consider maintenance for multiple units in an NPP in parallel,
  it has to be decided whether to accept a particular replacement and, in the positive
  case in which unit to conduct the corresponding replacement. In this case, the set \xmlNode{capitals}
  is used to provided indices for these units (see Section~\ref{subsec:Sets}).

  \item \textbf{Available Capitals}: This is available budgets for resources/units. The parameter
  \xmlNode{available\_capitals} is used to specify the available capitals for different
  resources/units at different $t$ (see Section~\ref{subsec:Parameters}).

  \item \textbf{Non-Selection}: Not selecting a project also has implications, inducing a growth
  in O\&M costs in future years, a decrease in plant production, an increase in forced outages,
  and even risking a premature end to plant life. Thus, not selecting a project can be seen as
  one more “option” for how a larger project is executed, expanding the list discussed earlier.
  Selection is of the “do nothing” option is reflected in both liability streams and reward
  streams. This can be activated by setting \xmlNode{nonSelection} to \xmlString{True}
  (see Section~\ref{subsec:Settings}).

  \item \textbf{Uncertainty}: One limitation of traditional optimization models for capital
  budgeting is that they do not account for uncertainty in reward and cost streams associated
  with individual projects, nor do they account for uncertainty in resource availability in
  future years. Projects can incur cost over-runs, especially when projects are large, performed
  infrequently, or when there is uncertainty regarding technical viability, external contractors,
  and/or suppliers of requisite parts and materials. Occasionally, projects are performed ahead
  of schedule and with savings in cost. Planned budgets for capital improvements can be cut, and key
  personnel may be lost. Or, there may be surprise budgetary windfalls for maintenance activities
  due to decreased costs for “unplanned” maintenance. The XML node \xmlNode{Uncertainties} is used
  to specify such uncertainties (see Section~\ref{subsec:Uncertainties}).

  %\item \textbf{Synergies}: Selecting a project may require replacing a structure, system, or
  %component (SSC) during a planned outage of the plant. Depending on the physical location of
  %an SSC in the plant and its relationship to other components, selecting one project may
  %reduce the cost of selecting another project (e.g., time or know-how required to implement
  %the project) if they are selected at the same time or close in proximity. For example, if
  %a plant has two units, selecting a project for one unit in a spring outage (e.g., replacement
  %of a condensate cooler and a set of feed-water heaters) may be followed by the same activity
  %in the fall outage in the second unit, at reduced cost.

  %\item \textbf{Planned Outage}: Nuclear power plants have planned outages at regular
  %intervals (e.g., every 18 months) often in the fall and spring to be well-prepared for winter
  %and summer peaks in load. While refueling only takes a fraction of a two-month (say) period
  %without power production, maintenance projects may be deferred until an outage. Moreover,
  %an outage can provide the only possible time period in which to carry out certain types of
  %projects. Because of lost revenue, an operator seeks to limit downtime. As a result, this
  %provides a special type of resource constraint limiting project selection due to multiple
  %projects competing for time, space, personnel, and equipment during an outage.

\end{itemize}

LOGOS consists of a collection of modeling entities/components that define different
aspects of the model, including \xmlNode{Sets}, \xmlNode{Parameters},
\xmlNode{Uncertainties}, and \xmlNode{ExternalConstraints}.
In addition, the \xmlNode{Setting} block specifies how the overall computation should run.

%
\subsection{Sets}
\label{subsec:Sets}

This subsection contains information regarding the XML nodes used to define the
\xmlNode{Sets} of the optimization model being performed through LOGOS.
\xmlNode{Sets} specifies a collection of data, possibly including
numeric data (e.g. real or integer values) as well as symbolic data (e.g. strings)
typically used to specify the valid indices for indexed components.
\nb Numeric data provided in \xmlNode{Sets} would be treated as strings.
\xmlNode{Sets} accepts the following additional sub-nodes:
\begin{itemize}
  \item \xmlNode{investments}, \xmlDesc{comma/space-separated string, required}, specifies
  the valid indices for investment projects.
  \item \xmlNode{capitals}, \xmlDesc{comma/space-separated string, optional},
  specifies the indices for NPP units.
  \item \xmlNode{time\_periods}, \xmlDesc{comma/space-separated string, optional},
  specifies the indices for time.
  \item \xmlNode{resources}, \xmlDesc{comma/space-separated string, optional},
  specifies indices for the resources and liabilities.
  \item \xmlNode{options}, \xmlDesc{semi-colon separated list of strings, optional},
  specifies the indices for multiple project-option pairs.
  This sub-node accepts the following attribute:
  \begin{itemize}
    \item \xmlAttr{index}, \xmlDesc{string, required}, specifies the index dependence.
    Valid index is \xmlString{investments}.
  \end{itemize}
\end{itemize}

Example XML:
\begin{lstlisting}[style=XML]
<Sets>
  <investments>
      HPFeedwaterHeaterUpgrade
      PresurizerReplacement
      ...
      ReplaceMoistureSeparatorReheater
  </investments>
  <time_periods>year1 year2 year3 year4 year5</time_periods>
  <resources>CapitalFunds OandMFunds</resources>
  <options index='investments'>
    PlanA PlanB DoNothing;
    PlanA PlanB PlanC;
    ...
    PlanA PlanB PlanC DoNothing
  </options>
</Sets>
\end{lstlisting}


%
\subsection{Parameters}
\label{subsec:Parameters}
This subsection contains information regarding the XML nodes used to define the
\xmlNode{Parameters} of the optimization model being performed through LOGOS:
\begin{itemize}
  \item \xmlNode{net\_present\_values}, \xmlDesc{comma/space-separated string, required},
  specifies the NPVs for capital projects or project-option pairs. This node accepts the
  following optional attribute:
  \begin{itemize}
    \item \xmlAttr{index}, \xmlDesc{comma-separated string, optional},
    specifies the indices of this parameter; keywords should be predefined in \xmlNode{Sets}.
    Valid keywords are \xmlString{investments} and \xmlString{options}.
    \default{investments}
  \end{itemize}
  \item \xmlNode{costs}, \xmlDesc{comma/space-separated string, required},
  specifies the costs for capital projects or project-option pairs. This node accepts the
  following optional attribute:
  \begin{itemize}
    \item \xmlAttr{index}, \xmlDesc{comma-separated string, optional},
    specifies the indices of this parameter; keywords should be predefined in \xmlNode{Sets}.
    Valid keywords are \xmlString{investments}, \xmlString{investments, time\_periods},
    \xmlString{options}, \xmlString{options, resources}, \xmlString{options, time\_periods},
    and \xmlString{options, resources, time\_periods}.
    \default{`investments'}
  \end{itemize}
  \item \xmlNode{available\_capitals}, \xmlDesc{comma/space-separated string, required},
  specifies the available capitals for capital projects or project-option pairs.
  This node accepts the following optional attribute:
  \begin{itemize}
    \item \xmlAttr{index}, \xmlDesc{comma-separated string, optional},
    specifies the indices of this parameter; keywords should be predefined in \xmlNode{Sets}.
    Valid keywords are \xmlString{resources}, \xmlString{time\_periods}, \xmlString{capitals},
    \xmlString{resources, time\_periods}, and \xmlString{capitals, time\_periods}
    \default{None}
  \end{itemize}
\end{itemize}

Example XML:
\begin{lstlisting}[style=XML]
<Parameters>
  <net_present_values index='options'>
    27.98 27.17 0.
    -10.07 -9.78 -9.22
    ...
    8.26 7.56 7.34 0.
  </net_present_values>
  <costs index='options,resources,time_periods'>
    12.99 1.3 0 0 0
    ...
    0.01 0 0 0 0
  </costs>
  <available_capitals index="resources,time_periods">
    22.6 36.7 20.6 23.6 22.7
    0.08 0.17 0.05 0.15 0.14
  </available_capitals>
</Parameters>
\end{lstlisting}


%
\subsection{Uncertainties}
\label{subsec:Uncertainties}
This subsection contains information regarding the XML nodes used to define the
\xmlNode{Uncertainties} of the optimization model being performed through LOGOS:
\begin{itemize}
  \item \xmlNode{available\_capitals}, \xmlDesc{optional}, specifies the scenarios
  associated with available capitals. This node accepts the attribute \xmlAttr{index}, which
  should be consistent with the \xmlNode{available\_capitals} defined in \xmlNode{Parameters}.
  This node accepts the following sub-nodes:
  \begin{itemize}
    \item \xmlNode{totalScenarios}, \xmlDesc{integer, required}, specifies the total
    number of scenarios for this parameter.
    \item \xmlNode{probabilities}, \xmlDesc{comma/space-separated float, required},
    specifies the probability for each scenario. The length should be equal to the total number of
    scenarios.
    \item \xmlNode{scenarios}, \xmlDesc{comma/space-separated float, required},
    specifies all scenarios for this parameter. The length should be equal to the total number
    of scenarios multiplied by the length of this parameter, as defined in \xmlNode{Parameters}.
  \end{itemize}

  \item \xmlNode{net\_present\_values}, \xmlDesc{optional}, specifies the scenarios
  associated with net\_present\_values. This node accepts the attribute \xmlAttr{index}, which
  should be consistent with the \xmlNode{net\_present\_values} defined in \xmlNode{Parameters}.
  \begin{itemize}
    \item \xmlNode{totalScenarios}, \xmlDesc{integer, required}, specifies the total
    number of scenarios for this parameter.
    \item \xmlNode{probabilities}, \xmlDesc{comma/space-separated float, required},
    specifies the probability for each scenario. The length should be equal to the total number of
    scenarios.
    \item \xmlNode{scenarios}, \xmlDesc{comma/space-separated float, required},
    specifies all scenarios for this parameter. The length should be equal to the total number
    of scenarios multiplied by the length of this parameter, as defined in \xmlNode{Parameters}.
  \end{itemize}
\end{itemize}

The overall number of scenarios is the total number of scenarios for \xmlNode{available\_capitals}
multiplied by the total number of scenarios for \xmlNode{net\_present\_values}.

Example XML:
\begin{lstlisting}[style=XML]
<Uncertainties>
  <available_capitals index="resources,time_periods">
    <totalScenarios>10</totalScenarios>
    <probabilities>
      0.5, 0.5
    </probabilities>
    <scenarios>
      20.0 34.0 17.0 20.0 18.0 0.08 0.17 0.05 0.15 0.14
      23.0 38.0 22.0 25.0 24.0 0.08 0.17 0.05 0.15 0.14
    </scenarios>
  </available_capitals>
  <net_present_values index='options'>
    <totalScenarios>9</totalScenarios>
    <probabilities>
      0.3 0.8
    </probabilities>
    <scenarios>
      13.3129 12.0228 0.0 -10.07
      ...
    </scenarios>
  </net_present_values>
</Uncertainties>
\end{lstlisting}


%
\subsection{External Constraints}
\label{subsec:ExternalConstraints}

This subsection contains information regarding the XML nodes used to define the
\xmlNode{ExternalConstraints} of the optimization model being performed through LOGOS.
This node accepts the following sub-node(s):
\begin{itemize}
  \item \xmlNode{constraint}, \xmlDesc{string, required}, specifies the external Python
  module file name along with its absolute or relative path. This external Python
  module contains the user-defined additional constraint.
  \nb If a relative path is specified, the code first checks relative to the working directory,
  then it checks with respect to the location of the input file. The working directory can be
  specified in \xmlNode{Settings} (see Section~\ref{subsec:Settings}). In addition, the extension
  `.py' is optional for the module file name that was inputted in this node.
  This sub-node also requires the following attribute:
  \begin{itemize}
    \item \xmlAttr{name}, \xmlDesc{string, required}, specifies the name of the constraint that will
    be added to the optimization problem.
  \end{itemize}
\end{itemize}

Example XML:
\begin{lstlisting}[style=XML]
<ExternalConstraints>
  <constraint name="con_I">externalConst</constraint>
  <constraint name="con_II">externalConstII.py</constraint>
</ExternalConstraints>
\end{lstlisting}

These constraints are Python modules, with a format automatically interpretable by
LOGOS. For example, users can define their own constraint, and the code will be embedded
and use the constraint as though it were an active part of the code itself.
The following provides an example of a user-defined external constraint:

Example Python Function:
\begin{lstlisting}[language=python]
# External constraint function
import numpy as np
import pyomo.environ as pyomo

def initialize():
  """
    Optional Method
    Optimization model parameters values can be updated/modified
    without directly accessing the optimization model.
    Value(s) will be updated in-place.
    @ In, None
    @ Out, updateDict, dict, {paramName:paramInfoDict},  where
      paramInfoDict contains {Indices:Values}
      Indices are parameter indices (either strings or tuples of
      strings, depending on whether there is one or
      more than one dimension). Values are the new values being
      assigned to the parameter at the given indices.
  """
  updateDict = {'available_capitals':{'None':16},
                'costs':{'1':1,'2':3,'3':7,'4':4,'5':8,
                         '6':9,'7':6,'8':10,'9':2,'10':5}
               }
  return updateDict

def constraint(var, sets, params):
  """
    Required Method
    External constraint provided by users that will be added to
    optimization problem
    @ In, sets, dict, all "Sets" provided in the Logos input
      file will be stored and available in this dictionary,
      i.e. {setName: setObject}
    @ In, params, dict, all "Parameters" provided in the
      Logos input file will be stored and
      available in this dictionary, i.e. {paramName: paramObject}
    @ In, var, object, the internally used decision variable,
      the dimensions/indices of this variable depend the type of
      optimization problems (i.e. "<problem_type>" from Logos
      input file). Currently, we will accept the following
      problem types:

      1. "singleknapsack": in this case, "var" will be var[:],
         where the index will be the element from
         xml node of "investment" in Logos input file.

      2. "multipleknapsack": in this case, "var" will be var[:,:],
         where the indices are the combinations element from set
         "investment" and element from set "capitals" in Logos
         input file

      3. "mckp": in this case, "var" will be var[:,:], where the
         indices are the combinations element from set
         "investment" and element from set "options" in Logos
         input file

      (Note that any element that is used as index will be
      converted to a string even if a number is provided in
      the Logos input file).

    @ Out, constraint, tuple, either (constraintRule,)
      or (constraintRule, indices)

    (Note that any modifications in provided sets and params
    will only have impact on this local module,
    i.e. the external constraint. In other words, the Sets
    and Params used in the internal constraints and
    objective will be kept unchanged!)
  """
  # All sets and parameters can be retrieved from dictionary
  # "sets" and "params" investments = sets['investments']

  def constraintRule(self, i):
    """
      Expression for user provided external constraint
      @ In, self, object, required to present, but not used
      @ In, i, str, element for the index set
      @ Out, constraintRule, function expression, expression
        to define user provided constraint

      Note that: Constraints can be indexed by lists or sets.
      When the return of function "constraint" contains
      lists or sets except the "constraintRule", the elements
      are iteratively passed to the rule function. If there
      is more than one, then the cross product is sent.
      For example, this constraint could be interpreted as
      placing limit on "ith" decision variable "var".
      A list of constraints for all "ith" decision variable
      "var" will be added to the optimization model
    """
    return var[i] <= 1

  # A tuple is required for the return, the first element
  # should be always the "constraintRule",
  # while the rest of elements are the lists or sets
  # if the user wants to construct the constraints
  # iteratively (See the docstring in "constraintRule"),
  # otherwise, keep it empty
  return (constraintRule, investments)
\end{lstlisting}

%
\subsection{Settings: Options for Optimization}
\label{subsec:Settings}

This subsection contains information regarding the XML nodes used to define the
\xmlNode{Settings} of the optimization model being performed through LOGOS:
\begin{itemize}
  \item \xmlNode{problem\_type}, \xmlDesc{string, required parameter}, specifies the type of
  optimization problem. Available types include \xmlString{SingleKnapsack},
  \xmlString{MultipleKnapsack}, and \xmlString{MCKP} for risk-informed stochastic optimization.
  Available types include \xmlString{droskp}, \xmlString{dromkp}, and \xmlString{dromckp}
  for distributionally robust optimization. Available types include \xmlString{cvarskp},
  \xmlString{cvarmkp}, and \xmlString{cvarmckp}.
  \item \xmlNode{solver}, \xmlDesc{string, optional parameter}, represents available solvers including
  \xmlNode{cbc} from \url{https://github.com/coin-or/Cbc.git} and \xmlNode{glpk} from
  \url{https://www.gnu.org/software/glpk/}.
  \item \xmlNode{sense}, \xmlDesc{string, optional parameter}, specifies \xmlString{minimize}
  or \xmlString{maximize} for minimization or maximization, respectively.
  \default{minimize}
  \item \xmlNode{mandatory}, \xmlDesc{comma/space-separated string, optional parameter},
  specifies regulatorily mandated or must-do projects.
  \item \xmlNode{nonSelection}, \xmlDesc{boolean, optional parameter}, indicates whether the
  investments options includes \textit{DoNothing} option.
  \default{False}
  \item \xmlNode{lowerBounds}, \xmlDesc{comma/space-separated integers, optional parameter}, specifies the lower bounds
  for decision variables.
  \item \xmlNode {upperBounds}, \xmlDesc{comma/space-separated integers, optional parameter}, specifies the upper bounds
  for decision variables.
  \item \xmlNode{consistentConstraintI}, \xmlDesc{string, optional parameter}, indicates whether
  this constraint is enabled.
  \default{True}
  \item \xmlNode{consistentConstraintII}, \xmlDesc{string, optional parameter}, indicates whether
  this constraint is enabled or not.
  \default{False}
  \item \xmlNode{solverOptions}, \xmlDesc{optional parameter}, accepts
  different options for the given solver provided in \xmlNode{solver}. A simple XML node only containing
  node tags and node texts can be used to provide the options for the solver. For example:
  \begin{lstlisting}[style=XML]
    <solverOptions>
      <threads>1</threads>
      <StochSolver>EF</StochSolver>
    </solverOptions>
  \end{lstlisting}
  In addition, if the problem type is distributionally robust optimization, additional option
  \xmlNode{radius\_ambiguity} can be used to control the Wasserstein distance. See Section~\ref{sec:DROCapitalBudgeting}.
  If the problem type is conditional value at risk (See Section~\ref{sec:CVaR}), additional options are available:
  \begin{itemize}
    \item \xmlNode{risk\_aversion}, \xmlDesc{float within $[0,1]$, optional parameter}, indicates the weight on
    maximizing expected NPV versus penalizing solutions that yield low-NPV scenarios.
    \item \xmlNode{confidence\_level}, \xmlDesc{float within $[0,1]$, optional parameter},  indicates
    the confidence level, i.e. the $\alpha$-percentile of the loss.
  \end{itemize}
\end{itemize}

Example XML:
\begin{lstlisting}[style=XML]
<Settings>
  <mandatory>
    PresurizerReplacement
    ...
    ReplaceInstrumentationAndControlCables
  </mandatory>
  <nonSelection>True</nonSelection>
  <consistentConstraintI>True</consistentConstraintI>
  <consistentConstraintII>True</consistentConstraintII>
  <solver>cbc</solver>
  <solverOptions>
    <threads>1</threads>
    <StochSolver>EF</StochSolver>
  </solverOptions>
  <sense>maximize</sense>
  <problem_type>mckp</problem_type>
</Settings>
\end{lstlisting}

    \input{include/DeterministicCapitalBudgeting.tex}
    \input{include/StochasticCapitalBudgeting.tex}
    \input{include/DRO.tex}
    \input{include/CVaR.tex}
    \input{include/PluginForRavenCode.tex}
    \input{include/CashflowAndNPVModels.tex}
    \section{Knapsack Models}
\label{sec:KnapsackModels}

LOGOS contains a set of Knapsack models that can be used coupled with RAVEN when 
the desired optimization problem requires the use of specific models to generate
knapsack required parameters.
More specifically, all these models would be contained in a RAVEN EnsembleModel 
and RAVEN optimization methods (e.g., Genetic Algorithms) would be employed to 
find the optimal solution.


\subsection{Simple Knapsack Model}
\label{subsec:SimpleKnapsackModel}
This model considers the classical Knapsack Model characterized by a set of elements 
that can be chosen (or not).
The goal is to maximize the sum of the chosen element values provided that the sum of 
element cost values satisfy capacity constraint (specified in the variable defined 
in the \xmlNode{capacity} node).

The ID of the variables that represent cost, value, and choice of each element are 
indicated in the \xmlNode{map} node.
The model generates two variables:
\begin{itemize}
  \item the validity of the chosen solution (specified in the \xmlNode{outcome} node): either 
        valid (i.e., 0), or invalid (i.e., 1) if the capacity constraint is not satisfied,
  \item totalValue (specified in the \xmlNode{choiceValue} node): sum of the values of the 
        chosen elements
\end{itemize}

When calculating the \xmlNode{choiceValue} variable, if the \xmlNode{capacity} constraint 
is not satisfied, then the \xmlNode{choiceValue} variable is penalized by multiplying the 
project value by -\xmlNode{penaltyFactor}.

Example LOGOS input XML for SimpleKnapsackModel Model:
\begin{lstlisting}[style=XML]
    <ExternalModel name="knapsack" subType="LOGOS.SimpleKnapsackModel">
      <variables>element1Status,element2Status,element3Status,
                 element4Status,element5Status,element1Val,
                 element2Val,element3Val,element4Val,
                 element5Val,element1Cost,element2Cost,
                 element3Cost,element4Cost,element5Cost,
                 validity,totalValue,capacityID</variables>
      <capacity>capacityID</capacity>
      <penaltyFactor>1.</penaltyFactor>
      <outcome>validity</outcome>
      <choiceValue>totalValue</choiceValue>
      <map value='element1Val'  cost='element1Cost' >element1Status</map>
      <map value='element2Val'  cost='element2Cost' >element2Status</map>
      <map value='element3Val'  cost='element3Cost' >element3Status</map>
      <map value='element4Val'  cost='element4Cost' >element4Status</map>
      <map value='element5Val'  cost='element5Cost' >element5Status</map>
    </ExternalModel>
\end{lstlisting}


\subsection{MultipleKnapsack Model}
\label{subsec:MultipleKnapsackModel}
This model considers the Multiple Knapsack Model characterized by a set of elements 
that can be chosen (or not) over a set of multiple knapsacks.
The goal is to maximize the sum of the chosen element values provided that the sum of 
element cost values satisfy capacity constraints of each knapsack.

The capacity of each knapsack is defined in the \xmlNode{knapsack} node.

The ID of the variables that represent cost, value, and choice of each element are 
indicated in the \xmlNode{map} node.
The model generates two variables:
\begin{itemize}
  \item the validity of the chosen solution (specified in the \xmlNode{outcome} node): either 
        valid (i.e., 0), or invalid (i.e., 1) if the capacity constraint is not satisfied,
  \item totalValue (specified in the \xmlNode{choiceValue} node): sum of the values of the 
        chosen elements
\end{itemize}

When calculating the \xmlNode{choiceValue} variable, if the capacity constraints 
are not satisfied, then the \xmlNode{choiceValue} variable is penalized by multiplying the 
project value by -\xmlNode{penaltyFactor}.

Example LOGOS input XML for MultipleKnapsack Model:
\begin{lstlisting}[style=XML]
  <Models>
    <ExternalModel name="knapsack" subType="LOGOS.MultipleKnapsackModel">
      <variables>e1Status,e2Status,e3Status,e4Status,e5Status,
                 e1Val   ,e2Val   ,e3Val   ,e4Val   ,e5Val,
                 e1Cost  ,e2Cost  ,e3Cost  ,e4Cost  ,e5Cost,
                 validity,totalValue,
                 K1_cap,K2_cap,K3_cap</variables>
      <knapsack ID='1'>K1_cap</knapsack>
      <knapsack ID='2'>K2_cap</knapsack>
      <knapsack ID='3'>K3_cap</knapsack>
      <penaltyFactor>1.</penaltyFactor>
      <outcome>validity</outcome>
      <choiceValue>totalValue</choiceValue>
      <map value='e1Val'  cost='e1Cost' >e1Status</map>
      <map value='e2Val'  cost='e2Cost' >e2Status</map>
      <map value='e3Val'  cost='e3Cost' >e3Status</map>
      <map value='e4Val'  cost='e4Cost' >e4Status</map>
      <map value='e5Val'  cost='e5Cost' >e5Status</map>
    </ExternalModel>
  </Models>
\end{lstlisting}


    \section*{Document Version Information}
    This document has been compiled using the following version of the plugin Git repository:
    \newline
    \input{../version.tex}

    % ---------------------------------------------------------------------- %
    % References
    %
    \clearpage
    % If hyperref is included, then \phantomsection is already defined.
    % If not, we need to define it.
    \providecommand*{\phantomsection}{}
    \phantomsection
    \addcontentsline{toc}{section}{References}
    \bibliographystyle{ieeetr}
    \bibliography{user_manual}


    % ---------------------------------------------------------------------- %

\end{document}
